% ------------------------------------------------------------------
\documentclass[12 pt]{article} % A4 paper set by geometry package below
\pagenumbering{arabic}
\setlength{\parindent}{10 mm}
\setlength{\parskip}{12 pt}

% Nimbus Sans font should be reasonably legible
\usepackage{helvet}
\renewcommand{\familydefault}{\sfdefault}
\usepackage[T1]{fontenc}  % Without this \textsterling produces $

% Section header spacing
\usepackage{titlesec}
\titlespacing\section{0pt}{12pt plus 4pt minus 2pt}{0pt plus 2pt minus 2pt}
\titlespacing\subsection{0pt}{12pt plus 4pt minus 2pt}{0pt plus 2pt minus 2pt}
\titlespacing\subsubsection{0pt}{12pt plus 4pt minus 2pt}{0pt plus 2pt minus 2pt}

\usepackage{amsmath}
\usepackage{amssymb}
\usepackage{graphicx}
\usepackage{verbatim}    % For comment
\usepackage[shortlabels]{enumitem}
\usepackage[paper=a4paper, marginparwidth=0 cm, marginparsep=0 cm, margin=2.5 cm, includemp]{geometry}
\usepackage[pdftex, pdfstartview={FitH}, pdfnewwindow=true, colorlinks=true, citecolor=blue, filecolor=blue, linkcolor=blue, urlcolor=blue, pdfpagemode=UseNone]{hyperref}

% Put module code and last-modified date in footer
\usepackage{fancyhdr}
\pagestyle{fancy}
\fancyhf{}
\renewcommand{\headrulewidth}{0pt}
\cfoot{{\small \thisweek}\hfill \thepage\hfill {\small \moddate}}

% Hopefully address Canvas complaints about pdf tagging and title
%\usepackage[tagged]{accessibility}
\hypersetup {
  pdfauthor={David Schaich},
  pdftitle={StatMech Homework},
}
% ------------------------------------------------------------------



% ------------------------------------------------------------------
% Shortcuts
\newcommand{\cO}{\ensuremath{\mathcal O} }
\newcommand{\be}{\ensuremath{\beta} }
\newcommand{\ga}{\ensuremath{\gamma} }
\newcommand{\ka}{\ensuremath{\kappa} }
\newcommand{\la}{\ensuremath{\lambda} }
\newcommand{\si}{\ensuremath{\sigma} }
\newcommand{\vdr}{\ensuremath{v_{\mathrm{dr}}} }
\renewcommand{\d}[1]{\ensuremath{\mathop{d#1}} }
\newcommand{\vev}[1]{\ensuremath{\left\langle #1 \right\rangle} }
\newcommand{\pderiv}[2]{\ensuremath{\frac{\partial #1}{\partial #2}} }
\newcommand{\showmarks}[1]{\rightline{\texttt{[#1 marks]}}} % \showmarks needs to follow a blank line!
% ------------------------------------------------------------------



% ------------------------------------------------------------------
\begin{document}
\newcommand{\thisweek}{MATH327 Homework 2}
\newcommand{\moddate}{Last modified 25 Mar.~2025}
\begin{center}
  {\Large \textbf{MATH327: StatMech and Thermo, Spring 2025}} \\[12 pt]
  {\Large \textbf{Second homework assignment}} \\[24 pt]
\end{center}

\section*{Instructions}
Complete all four questions below and submit your solutions by file upload \href{https://canvas.liverpool.ac.uk/courses/76365/assignments/297934}{on Canvas}.
Clear and neat presentations of your workings and the logic behind them will contribute to your mark.
Use of resources beyond the module materials must be explicitly referenced in your submissions.
This assignment is \textbf{due by 17:00 on Friday, 4 April 2025}.
Anonymous marking is turned on and I will aim to return feedback by Monday, 21 April 2025.

You should already be familiar with the Department's \href{https://canvas.liverpool.ac.uk/courses/76365/files/11992667}{academic integrity guidance} for 2024--2025, which states that by submitting solutions to this assessment you affirm that you have read and understood the \href{https://www.liverpool.ac.uk/media/livacuk/tqsd/code-of-practice-on-assessment/appendix_L_cop_assess.pdf}{Academic Integrity Policy} detailed in Appendix L of the Code of Practice on Assessment, and that you have successfully passed the Academic Integrity Tutorial and Quiz in the course of your studies.
You also affirm that the work you are submitting is your own and you have not commissioned production of the work from a third party or used artificial intelligence (AI) software \href{https://www.liverpool.ac.uk/media/livacuk/centre-for-innovation-in-education/digital-education/generative-ai-teach-learn-assess/guidance-on-the-use-of-generative-ai.pdf}{in an unacceptable manner} to generate the work.
(Generative AI software applications include, but are not limited to, ChatGPT, MS Copilot, and Gemini.)
You also affirm that you have not plagiarized material from another person or source, nor fabricated, falsified or embellished data when completing this assignment.
You also affirm that you have not colluded with any other student in the preparation or production of your work.
The marks achieved on this assessment remain provisional until they are ratified by the Board of Examiners in June 2025.
% ------------------------------------------------------------------



% ------------------------------------------------------------------
\newpage
\section*{Question 1: Diesel cycle}
Consider the Diesel cycle defined by the $PV$~diagram shown below, in which the `compression' stage $1 \to 2$ and the `power' stage $3 \to 4$ are both adiabatic, while the pressure is constant during the `injection/ignition' stage $2 \to 3$, and the volume is constant during the `exhaust' stage $4 \to 1$.
The compression ratio is $r \equiv V_1 / V_2 > 1$ and the cutoff ratio is $C \equiv V_3 / V_2 > 1$, with $C < r$.

\begin{center}\includegraphics[width=0.8\textwidth]{figs/Diesel.pdf}\end{center}

\begin{enumerate}[label={(\alph*)}]
  \item By computing $W_{\text{out}}$, $W_{\text{in}}$ and $Q_{\text{in}}$, show that the Diesel cycle's efficiency is
        \begin{equation*}
          \eta_D = 1 - \frac{f(C)}{r^{2 / 3}}
        \end{equation*}
        and determine the function $f(C)$ that depends only on the cutoff ratio.

  \showmarks{18}

  \item Show $f(C) > 1$ for $C > 1$.
        This indicates the Diesel cycle is less efficient than the Otto cycle with the same compression ratio $r$.

  \showmarks{6}
\end{enumerate}
% ------------------------------------------------------------------



% ------------------------------------------------------------------
\newpage
\section*{Question 2: Mixed ideal gases}
Consider a mixture of two classical ideal gases in thermodynamic equilibrium, in a container of volume $V$ at temperature $T$, like that illustrated below.
Let $N_1$ and $N_2$ be the particle numbers of the two gases.
Within each gas the particles are indistinguishable, but particles of one gas are distinguishable from particles of the other gas.
In particular, they have different masses $m_1$ and $m_2$, implying different thermal de~Broglie wavelengths $\la_1$ and $\la_2$. \\[-24 pt]
\begin{center}\includegraphics[width=0.3\textwidth]{figs/mixed.pdf}\end{center}
\vspace{-12 pt}

\begin{enumerate}[label={(\alph*)}]
  \item First consider the case of fixed particle numbers.
        Calculate the partition function $Z$ and the Helmholtz free energy of the ($N_1 + N_2$)-particle mixture, approximating $\log(N_i!) \approx N_i\log N_i - N_i$.

  \showmarks{4}

  \item Calculate the internal energy $\vev{E}$ and the entropy $S$ of the mixture.
        What is the condition of constant entropy?

  \showmarks{4}

  \item Calculate the pressure $P$ of the mixture, and relate it to the pressures $P_1$ and $P_2$ of each gas in isolation (as illustrated below).

  \showmarks{4}

  \item Now allow the particle numbers to fluctuate, with distinct chemical potentials $\mu_1$ and $\mu_2$ for each type of particle.
        Calculate the grand-canonical partition function $Z_g$ and the grand-canonical potential of the mixture, again approximating $\log(N_i!) \approx N_i\log N_i - N_i$.

  \showmarks{4}

  \item Calculate the average particle number $\vev{N} = \vev{N_1} + \vev{N_2}$.

  \showmarks{5}

  \item Re-calculate the internal energy $\vev{E}$ now that the particle numbers are allowed to fluctuate --- explicitly evaluate the relevant derivatives to ensure you obtain what you expect; don't just write down an answer by inspection.

  \showmarks{5}
\end{enumerate}

\begin{center}\includegraphics[width=0.3\textwidth]{figs/red.pdf}\hspace{0.3\textwidth}\includegraphics[width=0.3\textwidth]{figs/blue.pdf}\end{center}
% ------------------------------------------------------------------



% ------------------------------------------------------------------
\newpage
\section*{Question 3: Particle number fluctuations}
Consider the fugacity expansion of the grand-canonical partition function (Eq.~82 in the lecture notes):
\begin{equation*}
  Z_g(T, \mu) = \sum_{N = 0}^{\infty} \xi^N \, Z_N(T).
\end{equation*}
Here $\xi = e^{\be \mu} = e^{\mu / T}$ is the fugacity and $Z_N(T)$ is the $N$-particle canonical partition function, which is independent of $\xi$.
Recall that $\Phi(T, \mu) = -T \log Z_g(T, \mu)$ is the corresponding grand-canonical potential.

\begin{enumerate}[label={(\alph*)}]
  \item Derive a relation between the average particle number $\vev{N}$ and the derivative $\displaystyle \pderiv{}{\log \xi}\Phi$.

  \showmarks{8}

  \item Derive a relation between $\vev{\left(N - \vev{N}\right)^2}$ and $\displaystyle \left(\pderiv{}{\log \xi}\right)^2 \Phi$.

  \showmarks{8}

  \item Specialize to the case of Maxwell--Boltzmann statistics, for which the fugacity expansion simplifies to $Z_g^{\text{MB}}(T, \mu) = \exp[\xi Z_1(T)]$, where $Z_1$ is the single-particle partition function.
        Use the relations you have derived to determine $\vev{N}$ and $\vev{\left(N - \vev{N}\right)^2}$ for this case, and show
        \begin{equation*}
          \frac{\sqrt{\vev{\left(N - \vev{N}\right)^2}}}{\vev{N}} = \frac{1}{\sqrt{\vev{N}}}.
        \end{equation*}
        It may be useful to note $\displaystyle \pderiv{}{\log \xi} = \xi \pderiv{}{\xi}$.

  \showmarks{8}
\end{enumerate}
% ------------------------------------------------------------------



% ------------------------------------------------------------------
\newpage
\section*{Question 4: Grand-canonical massive bosons}
The grand-canonical potential for a gas of bosons with mass $m$ in a box of volume $V$, with temperature $T = 1 / \be$ and chemical potential $\mu$, is
\begin{equation*} % Spin zero, so only one polarization vs.\ the two for photons and spin-half fermions...
  \Phi(\be, \mu) = \frac{V}{\be} \frac{\sqrt{2m^3}}{2 \pi^2 \hbar^3} \int_0^{\infty} \log\left[1 - e^{-\be (E - \mu)}\right] \sqrt{E} \d{E}.
\end{equation*}
Consider the case of vanishing chemical potential, $\mu = 0$. % See Schroeder problem 7.75 for more...

\begin{enumerate}[label={(\alph*)}]
  \item Calculate the particle density $\displaystyle \frac{\vev{N}}{V}$.

  \showmarks{6}

  \item Calculate the internal energy density $\displaystyle \frac{\vev{E}}{V}$ to show $\vev{E} = \ga \vev{N} T$ and determine the constant $\ga$.

  \showmarks{7}

  \item Calculate the entropy of the gas.
        What is the condition of constant entropy?

  \showmarks{6}

  \item Calculate the pressure $P$ to show $P V = \ka \vev{N} T$ and determine the constant $\ka$.

  \showmarks{7}
\end{enumerate}

\noindent\textbf{Hints:} You can look up derivatives or integrals, citing any sources you use.
You can leave your answers in terms of the Riemann zeta function.
% ------------------------------------------------------------------



% ------------------------------------------------------------------
\end{document}
% ------------------------------------------------------------------
