% ------------------------------------------------------------------
\documentclass[12 pt]{article} % A4 paper set by geometry package below
\pagenumbering{arabic}
\setlength{\parindent}{10 mm}
\setlength{\parskip}{12 pt}

% Nimbus Sans font should be reasonably legible
\usepackage{helvet}
\renewcommand{\familydefault}{\sfdefault}
\usepackage[T1]{fontenc}  % Without this \textsterling produces $

% Section header spacing
\usepackage{titlesec}
\titlespacing\section{0pt}{12pt plus 4pt minus 2pt}{0pt plus 2pt minus 2pt}
\titlespacing\subsection{0pt}{12pt plus 4pt minus 2pt}{0pt plus 2pt minus 2pt}
\titlespacing\subsubsection{0pt}{12pt plus 4pt minus 2pt}{0pt plus 2pt minus 2pt}

\usepackage{amsmath}
\usepackage{amssymb}
\usepackage{graphicx}
\usepackage{verbatim}    % For comment
\usepackage[shortlabels]{enumitem}
\usepackage[paper=a4paper, marginparwidth=0 cm, marginparsep=0 cm, margin=2.5 cm, includemp]{geometry}
\usepackage[pdftex, pdfstartview={FitH}, pdfnewwindow=true, colorlinks=true, citecolor=blue, filecolor=blue, linkcolor=blue, urlcolor=blue, pdfpagemode=UseNone]{hyperref}

% Put module code and last-modified date in footer
\usepackage{fancyhdr}
\pagestyle{fancy}
\fancyhf{}
\renewcommand{\headrulewidth}{0pt}
\cfoot{{\small \thisweek}\hfill \thepage\hfill {\small \moddate}}

% Hopefully address Canvas complaints about pdf tagging and title
%\usepackage[tagged]{accessibility}
\hypersetup {
  pdfauthor={David Schaich},
  pdftitle={StatMech Homework},
}
% ------------------------------------------------------------------



% ------------------------------------------------------------------
% Shortcuts
\newcommand{\cO}{\ensuremath{\mathcal O} }
\newcommand{\be}{\ensuremath{\beta} }
\newcommand{\De}{\ensuremath{\Delta} }
\newcommand{\eps}{\ensuremath{\varepsilon} }
\newcommand{\si}{\ensuremath{\sigma} }
\newcommand{\vdr}{\ensuremath{v_{\mathrm{dr}}} }
\renewcommand{\d}[1]{\ensuremath{\mathop{d#1}} }
\newcommand{\vev}[1]{\ensuremath{\left\langle #1 \right\rangle} }
\newcommand{\pderiv}[2]{\ensuremath{\frac{\partial #1}{\partial #2}} }
\newcommand{\showmarks}[1]{\rightline{\texttt{[#1 marks]}}} % \showmarks needs to follow a blank line!
% ------------------------------------------------------------------



% ------------------------------------------------------------------
\begin{document}
\newcommand{\thisweek}{MATH327 Homework 1}
\newcommand{\moddate}{Last modified 26 Feb.~2025}
\begin{center}
  {\Large \textbf{MATH327: StatMech and Thermo, Spring 2025}} \\[12 pt]
  {\Large \textbf{First homework assignment}} \\[24 pt]
\end{center}

\section*{Instructions}
Complete all four questions below and submit your solutions by file upload \href{https://canvas.liverpool.ac.uk/courses/76365/assignments/297933}{on Canvas}.
Clear and neat presentations of your workings and the logic behind them will contribute to your mark.
Use of resources beyond the module materials must be explicitly referenced in your submissions.
This assignment is \textbf{due by 17:00 on Friday, 7 March 2025}.
Anonymous marking is turned on and I will aim to return feedback by Monday, 25 March 2025.

You should already be familiar with the Department's \href{https://canvas.liverpool.ac.uk/courses/76365/files/11992667}{academic integrity guidance} for 2024--2025, which states that by submitting solutions to this assessment you affirm that you have read and understood the \href{https://www.liverpool.ac.uk/media/livacuk/tqsd/code-of-practice-on-assessment/appendix_L_cop_assess.pdf}{Academic Integrity Policy} detailed in Appendix L of the Code of Practice on Assessment, and that you have successfully passed the Academic Integrity Tutorial and Quiz in the course of your studies.
You also affirm that the work you are submitting is your own and you have not commissioned production of the work from a third party or used artificial intelligence (AI) software \href{https://www.liverpool.ac.uk/media/livacuk/centre-for-innovation-in-education/digital-education/generative-ai-teach-learn-assess/guidance-on-the-use-of-generative-ai.pdf}{in an unacceptable manner} to generate the work.
(Generative AI software applications include, but are not limited to, ChatGPT, MS Copilot, and Gemini.)
You also affirm that you have not plagiarized material from another person or source, nor fabricated, falsified or embellished data when completing this assignment.
You also affirm that you have not colluded with any other student in the preparation or production of your work.
The marks achieved on this assessment remain provisional until they are ratified by the Board of Examiners in June 2025.
% ------------------------------------------------------------------



% ------------------------------------------------------------------
\vfill
\section*{Question 1: Drift and diffusion}
At 06:30~UTC on Monday, 28 April 1986, radiation detectors started going off at the Forsmark Nuclear Power Plant in Sweden.
Within a few hours, the Swedes had confirmed that the radiation was coming from some distant source (rather than Forsmark itself).
Based on the direction of the wind, the Swedish government asked the USSR what had happened.
Although the Soviet authorities initially denied there had been any incident, that evening they released a \href{https://www.youtube.com/watch?v=sC7n_QgJRks}{15-second news bulletin} reporting an accident at the Chernobyl Nuclear Power Plant in northern Ukraine.

The lack of reliable official information made it urgent to estimate how severe this accident may have been.
We can do this by modelling the motion of each radioactive particle in the atmosphere as a one-dimensional random walk along the $1100~\mathrm{km}$ line between Chernobyl and Forsmark.

\newpage
Suppose that the wind produced a steady drift velocity $\vdr = 14~\mathrm{km}/\mathrm{hour}$ from Chernobyl to Forsmark, and that the radioactive particles have a diffusion constant $D = 7.3~\mathrm{km}/\sqrt{\mathrm{hour}}$.
Further suppose that measurements at 06:30~UTC indicated a billion becquerels ($1~\mathrm{GBq}$) of radioactivity had travelled at least $1100~\mathrm{km}$ from Chernobyl, while further measurements at 09:30~UTC indicated rapid growth in this radioactivity, to $100~\mathrm{GBq}$.
(The \href{https://en.wikipedia.org/wiki/Becquerel}{becquerel} is the SI unit of radioactivity.)

\begin{enumerate}[label={(\alph*)}]
  \item Using the central limit theorem, show that the accident happened around 21:30 UTC on Friday, 25 April 1986 (nearly three days before it was publicly acknowledged).
        This is easiest to do numerically, for example using the Python-based hints below.

  \showmarks{10}

  \item Even without completing part~(a), you can take it as given that approximately 57~hours passed between the accident and the first radiation detection at Forsmark.
        Use this to estimate how much radioactivity was released in the accident.

  \showmarks{10}

  \item The estimates above involve several simplifying assumptions and approximations.
        Choose at least one simplification and explain --- without attempting to carry out a corrected calculation --- whether it causes an underestimate or an overestimate of the amount of radioactive material released.

  \showmarks{6}
\end{enumerate}

\vfill
\noindent \textbf{Hints:} The error function
\begin{equation*}
  \mathrm{erf}(u) = \frac{1}{\sqrt{\pi}} \int_{-u}^u e^{-x^2} \d{x} = \frac{2}{\sqrt{\pi}} \int_0^u e^{-x^2} \d{x}, \qquad u \geq 0,
\end{equation*}
or its complement $\mathrm{erfc}(u) = 1 - \mathrm{erf}(u)$, may appear in your work.
The Python code below illustrates one possible numerical method to find the value of $u$ that satisfies the relation $\mathrm{erfc}(u) - 10^6 \mathrm{erfc}(u + 2) = 0$.
It uses a function provided by the \href{https://scipy.org}{SciPy} package, which takes three arguments:
The expression to set equal to zero (defined just above it), an initial guess, and (optionally) the method to use (here the \href{https://en.wikipedia.org/wiki/Levenberg-Marquardt_algorithm}{Levenberg--Marquardt algorithm}).
It's wise to check the result: \\[-30 pt]
\begin{verbatim}
>>> import numpy as np
>>> from scipy import special, optimize
>>> def diff(u):
...   return (special.erfc(u) - 1e6 * special.erfc(u + 2.0))
...
>>> all_out = optimize.root(diff, 2.0, method='lm')
>>> u = all_out.x[0]
>>> print("u = %.4g" % u)
u = 2.311
>>> a = special.erfc(u)
>>> b = 1e6 * special.erfc(u + 2.0)
>>> print("Check: %.8g = %.8g" % (a, b))
Check: 0.001081093 = 0.001081093
\end{verbatim}
\newpage
\noindent SciPy can also be used to evaluate $\mathrm{erfc}(u)$ for any $u \geq 0$ of interest: \\[-30 pt]
\begin{verbatim}
>>> import numpy as np
>>> from scipy import special
>>> for u in np.arange(2.0046779, 5.0, 0.3116608):
...   print("erfc(%.8g) = %.4g" % (u, special.erfc(u)))
...
erfc(2.0046779) = 0.004582
erfc(2.3163387) = 0.001054
erfc(2.6279995) = 0.000202
erfc(2.9396603) = 3.22e-05
erfc(3.2513211) = 4.264e-06
erfc(3.5629819) = 4.684e-07
erfc(3.8746427) = 4.264e-08
erfc(4.1863035) = 3.213e-09
erfc(4.4979643) = 2.003e-10
erfc(4.8096251) = 1.033e-11
\end{verbatim}
% ------------------------------------------------------------------



% ------------------------------------------------------------------
\vfill
\section*{Question 2: Negative temperature}
Consider a system of $N$ distinguishable particles in which the energy of each particle can take only two distinct values, $0$ and $\eps > 0$.
Denote by $n_0$ the number of particles that have energy $0$, and by $n_1 = N - n_0$ the number of particles that have energy $\eps$.
Assume the system is in thermodynamic equilibrium with both $n_0 \gg 1$ and $n_1 \gg 1$.

Suppose the system is isolated and governed by the micro-canonical ensemble with conserved internal energy $E$.

\begin{enumerate}[label={(\alph*)}]
  \item Approximating $\log(n!) \approx n\log n - n$, what is the entropy of the system in terms of $N$, $E$ and $\eps$?

  \showmarks{8}

  \item What is the temperature $T$ of the system in terms of $N$, $E$ and $\eps$?
        Show that the temperature is negative when $\frac{N\eps}{E} < C$ and determine the constant $C$.

  \showmarks{8}

  \item Explain what happens when a negative-temperature system is brought into thermal contact with a positive-temperature system.

  \showmarks{8}
\end{enumerate}
% ------------------------------------------------------------------



% ------------------------------------------------------------------
\newpage
\section*{Question 3: Heat capacity}
\begin{enumerate}[label={(\alph*)}]
  \item Starting from the expectation value for the internal energy in the canonical ensemble,
        \begin{equation*}
          \vev{E} = \frac{1}{Z} \sum_{i = 1}^M E_i \, e^{-\be E_i},
        \end{equation*}
        derive a relation between the heat capacity
        \begin{equation*}
          c_v = \pderiv{}{T} \vev{E}
        \end{equation*}
        and the quantity $\vev{\left(E - \vev{E}\right)^2}$.

  \showmarks{10}

  \item If the heat capacity vanishes at a non-zero temperature, what can we conclude about the micro-state energies $E_i$?

  \showmarks{4}

  \item Derive a relation between $c_v$, $\pderiv{}{T} c_v$ and the quantity $\vev{\left(E - \vev{E}\right)^3}$.

  \showmarks{10}
\end{enumerate}
% ------------------------------------------------------------------



% ------------------------------------------------------------------
\newpage
\section*{Question 4: Energy and entropy} % Could plug in specific N for variation from year to year...
Using the canonical ensemble with inverse temperature $\be = 1 / T$, consider a system of $N \gg 1$ indistinguishable non-interacting spins in an external magnetic field with strength $H > 0$.

\begin{enumerate}[label={(\alph*)}]
  \item What are the internal energy $\vev{E}_I$ and the entropy $S_I$ as functions of $\be$, $N$ and $H$?

  \showmarks{4}

  \item Use your results to confirm the following low- and high-temperature limits:
        \begin{align*}
               \lim_{T \to 0} \vev{E}_I & = -NH &      \lim_{T \to 0} S_I & = 0           \\
          \lim_{T \to \infty} \vev{E}_I & = 0   & \lim_{T \to \infty} S_I & = \log(N + 1)
        \end{align*}

  \showmarks{6}

  \item For low but non-zero temperatures, expand both $\vev{E}_I$ and $S_I$ in terms of $\eps \equiv e^{-2\be H} \ll 1$.
        Show that the largest temperature-dependent term in the energy expansion is proportional to $\eps$ and find the constant of proportionality.
        Show that the largest temperature-dependent term in the entropy expansion is proportional to $\be\eps$ and find the constant of proportionality.

  \showmarks{8}

  \item For high but finite temperatures, expand both $\vev{E}_I$ and $S_I$ in terms of $x \equiv 2\be H \ll 1$.
        Show that the largest temperature-dependent term in the energy expansion is proportional to $x$ and find the constant of proportionality.
        Show that the largest temperature-dependent term in the entropy expansion is proportional to $x^2$ and find the constant of proportionality.

  \showmarks{8}
\end{enumerate}

\noindent The following famous results may be useful:
\begin{align*}
  \frac{1}{1 - e^{-x}} & = \frac{1}{x} + \frac{1}{2} + \frac{x}{12} - \frac{x^3}{720} + \frac{x^5}{30\,240} + \cO\!\left(x^6\right) \\
  \log\left[1 - e^{-x}\right] & = \log(x) - \frac{x}{2} + \frac{x^2}{24} - \frac{x^4}{2880} + \frac{x^6}{181\,440} + \cO\!\left(x^7\right).
\end{align*}
% ------------------------------------------------------------------



% ------------------------------------------------------------------
\end{document}
% ------------------------------------------------------------------
