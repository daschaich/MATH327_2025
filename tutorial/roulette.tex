% ------------------------------------------------------------------
\documentclass[12 pt]{article} % A4 paper set by geometry package below
\pagenumbering{arabic}
\setlength{\parindent}{10 mm}
\setlength{\parskip}{12 pt}

% Nimbus Sans font should be reasonably legible
\usepackage{helvet}
\renewcommand{\familydefault}{\sfdefault}
\usepackage[T1]{fontenc}  % Without this \textsterling produces $

% Section header spacing
\usepackage{titlesec}
\titlespacing\section{0pt}{12pt plus 4pt minus 2pt}{0pt plus 2pt minus 2pt}
\titlespacing\subsection{0pt}{12pt plus 4pt minus 2pt}{0pt plus 2pt minus 2pt}
\titlespacing\subsubsection{0pt}{12pt plus 4pt minus 2pt}{0pt plus 2pt minus 2pt}

\usepackage{amsmath}
\usepackage{amssymb}
\usepackage{graphicx}
\usepackage{verbatim}    % For comment
\usepackage[paper=a4paper, marginparwidth=0 cm, marginparsep=0 cm, margin=2.5 cm, includemp]{geometry}
\usepackage[pdftex, pdfstartview={FitH}, pdfnewwindow=true, colorlinks=true, citecolor=blue, filecolor=blue, linkcolor=blue, urlcolor=blue, pdfpagemode=UseNone]{hyperref}

% Put module code and last-modified date in footer
\usepackage{fancyhdr}
\pagestyle{fancy}
\fancyhf{}
\renewcommand{\headrulewidth}{0pt}
\cfoot{{\small \thisunit}\hfill \thepage\hfill {\small \moddate}}

% Hopefully address Canvas complaints about pdf tagging
%\usepackage[tagged]{accessibility}
\hypersetup {
  pdfauthor={David Schaich},
  pdftitle={StatMech Tutorial},
}
% ------------------------------------------------------------------



% ------------------------------------------------------------------
% Shortcuts
\newcommand{\Rbb}{\ensuremath{\mathbb R} }
\newcommand{\De}{\ensuremath{\Delta} }
% ------------------------------------------------------------------



% ------------------------------------------------------------------
\begin{document}
\newcommand{\thisunit}{MATH327 Tutorial (Probabilities)}
\newcommand{\moddate}{Last modified 29 Jan.~2025}
\begin{center}
  {\Large \textbf{MATH327: StatMech and Thermo, Spring 2025}} \\[12 pt]
  {\Large \textbf{Tutorial problem \ --- \ Probabilities}} \\[24 pt]
\end{center}

This activity will be introduced in our 30 January tutorial, and while we'll probably review the first few parts that same day, you'll have until our 5 February lecture to work on the final part.
We'll begin the tutorial by introducing the roulette wheel (page 13 of the lecture notes), then we'll consider a simple game of roulette in which we just place bets on the colour of the pocket where the ball ends up (assuming a fair wheel).
If we bet correctly we get back twice the money we put in; otherwise we lose our money.
We define our (potentially negative) \textit{gain} to be the amount we receive minus the amount we spend on bets. \\[-20 pt]

\begin{itemize}
  \item Suppose we place \textsterling5 bets on `black' for each of $N$ spins of the roulette wheel.
        What are the probabilities and gains of winning and of losing for any single one of those spins?
        Letting $W = 0, \cdots, N$ be the number of spins where we win, what is our total gain $G_W$ as a function of $(N, W)$? \\[8 pt]
  \item The number of different ways we could win $W$ times out of $N$ attempts (in any order) is given by the binomial coefficient
        \begin{equation*}
          \binom{N}{W} = \frac{N!}{W! \; (N - W)!},
        \end{equation*}
        with $0! = 1$.
        Setting $N = 5$, what are the six probabilities $p_0$ through $p_5$ that we win $W = 0, \cdots, 5$ times?
        What is the general $p_W$ for any $(N, W)$? \\[8 pt]
  \item Now let's apply the central limit theorem.
        What are the mean gain and its variance for a single spin of the wheel?
        What is the resulting probability distribution $p(g)$ given by the central limit theorem for the gain $g \in \Rbb$ after $N \gg 1$ spins? \\[8 pt]
  \item In order to compare the \textbf{distribution} $p(g)$ against the probabilities $p_W$ considered above, we need to integrate $p(g)$ over appropriate intervals as discussed in class (and on page 16 of the lecture notes).
        Natural intervals to consider are
        \begin{equation*}
          P_{\text{integ}}(G_W) \equiv \int_{G_W - \De G / 2}^{G_W + \De G / 2} p(g) \; dg,
        \end{equation*}
        where $\De G = G_{W + 1} - G_W$ is a constant you can read off from your work so far.
        These numerical integrations are not convenient to do by hand, but can easily be performed by Maple, Python, MATLAB, Mathematica, etc.
        Alternatively, we can simplify further by approximating $p(g)$ as a constant within each interval, which would give us
        \begin{equation*}
          P_{\text{const}}(G_W) \equiv p(G_W) \; \De G.
        \end{equation*}
        Keeping $N = 5$, what are the six $P_{\text{integ}}$ and the six $P_{\text{const}}$?
        How do they compare to the exact $p_W$?
        How does the comparison change as $N$ increases?
\end{itemize}

\end{document}
% ------------------------------------------------------------------
