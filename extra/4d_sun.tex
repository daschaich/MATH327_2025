% ------------------------------------------------------------------
\documentclass[12 pt]{article} % A4 paper set by geometry package below
\pagenumbering{arabic}
\setlength{\parindent}{10 mm}
\setlength{\parskip}{12 pt}

% Nimbus Sans font should be reasonably legible
\usepackage{helvet}
\renewcommand{\familydefault}{\sfdefault}
\usepackage[T1]{fontenc}  % Without this \textsterling produces $

% Section header spacing
\usepackage{titlesec}
\titlespacing\section{0pt}{12pt plus 4pt minus 2pt}{0pt plus 2pt minus 2pt}
\titlespacing\subsection{0pt}{12pt plus 4pt minus 2pt}{0pt plus 2pt minus 2pt}
\titlespacing\subsubsection{0pt}{12pt plus 4pt minus 2pt}{0pt plus 2pt minus 2pt}

\usepackage{amsmath}
\usepackage{amssymb}
\usepackage{graphicx}
\usepackage{verbatim}    % For comment
\usepackage[shortlabels]{enumitem}
\usepackage[paper=a4paper, marginparwidth=0 cm, marginparsep=0 cm, margin=2.5 cm, includemp]{geometry}
\usepackage[pdftex, pdfstartview={FitH}, pdfnewwindow=true, colorlinks=true, citecolor=blue, filecolor=blue, linkcolor=blue, urlcolor=blue, pdfpagemode=UseNone]{hyperref}

% Put module code and last-modified date in footer
\usepackage{fancyhdr}
\pagestyle{fancy}
\fancyhf{}
\renewcommand{\headrulewidth}{0pt}
\cfoot{{\small \thisunit}\hfill \thepage\hfill {\small \moddate}}

% Hopefully address Canvas complaints about pdf tagging
%\usepackage[tagged]{accessibility}
\hypersetup {
  pdfauthor={David Schaich},
  pdftitle={StatMech Extra Practice},
}
% ------------------------------------------------------------------



% ------------------------------------------------------------------
\begin{document}
\newcommand{\thisunit}{MATH327 Extra (4d sun)}
\newcommand{\moddate}{Last modified 26 Apr.~2025}
\begin{center}
  {\Large \textbf{MATH327: StatMech and Thermo, Spring 2025}} \\[12 pt]
  {\Large \textbf{Extra practice \ --- \ 4d sun}} \\[24 pt]
\end{center}

What is the colour of the sun in four dimensions?

In more detail: Consider a universe with four spatial dimensions (in addition to the time dimension).
Derive the analogue of the Planck spectrum and find the wavelength where this is maximized as a function of temperature.
Then insert the surface temperature of the sun, $T \approx 5778$~K, and use the unit conversion factors below to put the wavelength in units of nm: \\[-24 pt]

$k_B \approx 8.617\times 10^{-5}$~eV$\cdot$K$^{-1}$ \\[-24 pt]

$\hbar \approx 6.582\times 10^{-16}$~eV$\cdot$s \\[-24 pt]

$c \approx 2.998\times 10^8$~m$\cdot$s$^{-1}$

As a warm-up, you can check that you're able to get the correct peak wavelength (roughly 500~nm) for our 3d sun, and an infrared wavelength (roughly 10~$\mu$m) for a warm-blooded animal.

\end{document}
% ------------------------------------------------------------------
