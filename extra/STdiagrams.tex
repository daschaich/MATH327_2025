% ------------------------------------------------------------------
\documentclass[12 pt]{article} % A4 paper set by geometry package below
\pagenumbering{arabic}
\setlength{\parindent}{10 mm}
\setlength{\parskip}{12 pt}

% Nimbus Sans font should be reasonably legible
\usepackage{helvet}
\renewcommand{\familydefault}{\sfdefault}
\usepackage[T1]{fontenc}  % Without this \textsterling produces $

% Section header spacing
\usepackage{titlesec}
\titlespacing\section{0pt}{12pt plus 4pt minus 2pt}{0pt plus 2pt minus 2pt}
\titlespacing\subsection{0pt}{12pt plus 4pt minus 2pt}{0pt plus 2pt minus 2pt}
\titlespacing\subsubsection{0pt}{12pt plus 4pt minus 2pt}{0pt plus 2pt minus 2pt}

\usepackage{amsmath}
\usepackage{amssymb}
\usepackage{graphicx}
\usepackage{verbatim}    % For comment
\usepackage[shortlabels]{enumitem}
\usepackage[paper=a4paper, marginparwidth=0 cm, marginparsep=0 cm, margin=2.5 cm, includemp]{geometry}
\usepackage[pdftex, pdfstartview={FitH}, pdfnewwindow=true, colorlinks=true, citecolor=blue, filecolor=blue, linkcolor=blue, urlcolor=blue, pdfpagemode=UseNone]{hyperref}

% Put module code and last-modified date in footer
\usepackage{fancyhdr}
\pagestyle{fancy}
\fancyhf{}
\renewcommand{\headrulewidth}{0pt}
\cfoot{{\small \thisunit}\hfill \thepage\hfill {\small \moddate}}

% Hopefully address Canvas complaints about pdf tagging
%\usepackage[tagged]{accessibility}
\hypersetup {
  pdfauthor={David Schaich},
  pdftitle={StatMech Extra Practice},
}
% ------------------------------------------------------------------



% ------------------------------------------------------------------
\begin{document}
\newcommand{\thisunit}{MATH327 Extra ($ST$~diagrams)}
\newcommand{\moddate}{Last modified 26 Apr.~2025}
\begin{center}
  {\Large \textbf{MATH327: StatMech and Thermo, Spring 2025}} \\[12 pt]
  {\Large \textbf{Extra practice \ --- \ $ST$ diagrams}} \\[24 pt]
\end{center}

We have extensively used pressure--volume ($PV$) diagrams as a way to visualize and analyse thermodynamic processes and cycles, taking advantage of the fact that these two quantities suffice to describe the complete thermodynamic macro-state of an ideal gas.
It is also possible to draw diagrams using different pairs of quantities, which in some cases can be more revealing.
After $PV$~diagrams, entropy--temperature ($ST$) diagrams are the next-most-common choice, with entropy on the horizontal axis and temperature on the vertical axis.

\begin{enumerate}[label={(\alph*)}]
  \item Draw an isotherm and an adiabat on an $ST$~diagram.

  \item The area underneath a process on a $PV$~diagram is the (negative) work done on the gas through that process.
        What does the area under a process on an $ST$~diagram correspond to?

  \item Draw the $ST$~diagram for the Carnot cycle.

  \item Draw $ST$~diagrams for the Otto and Diesel cycles.
\end{enumerate}
\end{document}
% ------------------------------------------------------------------
