% ------------------------------------------------------------------
\documentclass[12 pt]{article} % A4 paper set by geometry package below
\pagenumbering{arabic}
\setlength{\parindent}{10 mm}
\setlength{\parskip}{12 pt}

% Nimbus Sans font should be reasonably legible
\usepackage{helvet}
\renewcommand{\familydefault}{\sfdefault}
\usepackage[T1]{fontenc}  % Without this \textsterling produces $

% Section header spacing
\usepackage{titlesec}
\titlespacing\section{0pt}{12pt plus 4pt minus 2pt}{0pt plus 2pt minus 2pt}
\titlespacing\subsection{0pt}{12pt plus 4pt minus 2pt}{0pt plus 2pt minus 2pt}
\titlespacing\subsubsection{0pt}{12pt plus 4pt minus 2pt}{0pt plus 2pt minus 2pt}

\usepackage{amsmath}
\usepackage{amssymb}
\usepackage{graphicx}
\usepackage{verbatim}    % For comment
\usepackage[shortlabels]{enumitem}
\usepackage[paper=a4paper, marginparwidth=0 cm, marginparsep=0 cm, margin=2.5 cm, includemp]{geometry}
\usepackage[pdftex, pdfstartview={FitH}, pdfnewwindow=true, colorlinks=true, citecolor=blue, filecolor=blue, linkcolor=blue, urlcolor=blue, pdfpagemode=UseNone]{hyperref}

% Put module code and last-modified date in footer
\usepackage{fancyhdr}
\pagestyle{fancy}
\fancyhf{}
\renewcommand{\headrulewidth}{0pt}
\cfoot{{\small \thisunit}\hfill \thepage\hfill {\small \moddate}}

% Hopefully address Canvas complaints about pdf tagging
%\usepackage[tagged]{accessibility}
\hypersetup {
  pdfauthor={David Schaich},
  pdftitle={StatMech Extra Practice},
}
% ------------------------------------------------------------------



% ------------------------------------------------------------------
% Shortcuts
\newcommand{\cG}{\ensuremath{\mathcal G} }
\newcommand{\si}{\ensuremath{\sigma} }
\renewcommand{\d}[1]{\ensuremath{\mathop{d#1}} }
% ------------------------------------------------------------------



% ------------------------------------------------------------------
\begin{document}
\newcommand{\thisunit}{MATH327 Extra (Gaussian int.)}
\newcommand{\moddate}{Last modified 25 Jan.~2025}
\begin{center}
  {\Large \textbf{MATH327: StatMech and Thermo, Spring 2025}} \\[12 pt]
  {\Large \textbf{Extra practice \ --- \ Gaussian integrals}} \\[24 pt]
\end{center}

As mentioned in the module's \href{https://canvas.liverpool.ac.uk/courses/76365/files/12029982}{logistical information}, I anticipate that you have previously learned about \href{https://en.wikipedia.org/wiki/Gaussian_integral}{gaussian integrals}.
However, several students struggled with these last year, so a review may be helpful.

First show
\begin{equation*}
  \cG \equiv \int_{-\infty}^{\infty} e^{-x^2} \d{x} = \sqrt{\pi}.
\end{equation*}
There are several ways this can be done.  My favourite is to evaluate $\cG^2$ using polar coordinates.

Next use a change of variables to show
\begin{equation*}
  \int_{-\infty}^{\infty} e^{-a (x + b)^2} \d{x} = \sqrt{\frac{\pi}{a}},
\end{equation*}
assuming $a > 0$.

Finally consider the gaussian probability distribution
\begin{equation*}
  p(x) = C \exp\left[-\frac{(x - N\mu)^2}{2N\si^2}\right],
\end{equation*}
with finite $\mu$ and $\si^2 > 0$.
Based on the properties of probabilities, determine the positive coefficient $C$ and compare your result against Eq.~10 in the lecture notes.

\end{document}
% ------------------------------------------------------------------
