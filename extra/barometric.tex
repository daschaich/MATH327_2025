% ------------------------------------------------------------------
\documentclass[12 pt]{article} % A4 paper set by geometry package below
\pagenumbering{arabic}
\setlength{\parindent}{10 mm}
\setlength{\parskip}{12 pt}

% Nimbus Sans font should be reasonably legible
\usepackage{helvet}
\renewcommand{\familydefault}{\sfdefault}
\usepackage[T1]{fontenc}  % Without this \textsterling produces $

% Section header spacing
\usepackage{titlesec}
\titlespacing\section{0pt}{12pt plus 4pt minus 2pt}{0pt plus 2pt minus 2pt}
\titlespacing\subsection{0pt}{12pt plus 4pt minus 2pt}{0pt plus 2pt minus 2pt}
\titlespacing\subsubsection{0pt}{12pt plus 4pt minus 2pt}{0pt plus 2pt minus 2pt}

\usepackage{amsmath}
\usepackage{amssymb}
\usepackage{graphicx}
\usepackage{verbatim}    % For comment
\usepackage[shortlabels]{enumitem}
\usepackage[paper=a4paper, marginparwidth=0 cm, marginparsep=0 cm, margin=2.5 cm, includemp]{geometry}
\usepackage[pdftex, pdfstartview={FitH}, pdfnewwindow=true, colorlinks=true, citecolor=blue, filecolor=blue, linkcolor=blue, urlcolor=blue, pdfpagemode=UseNone]{hyperref}

% Put module code and last-modified date in footer
\usepackage{fancyhdr}
\pagestyle{fancy}
\fancyhf{}
\renewcommand{\headrulewidth}{0pt}
\cfoot{{\small \thisunit}\hfill \thepage\hfill {\small \moddate}}

% Hopefully address Canvas complaints about pdf tagging
%\usepackage[tagged]{accessibility}
\hypersetup {
  pdfauthor={David Schaich},
  pdftitle={StatMech Extra Practice},
}
% ------------------------------------------------------------------



% ------------------------------------------------------------------
\begin{document}
\newcommand{\thisunit}{MATH327 Extra (Barom.~eq.)}
\newcommand{\moddate}{Last modified 27 Apr.~2025}
\begin{center}
  {\Large \textbf{MATH327: StatMech and Thermo, Spring 2025}} \\[12 pt]
  {\Large \textbf{Extra practice \ --- \ Barometric equation}} \\[24 pt]
\end{center}

Consider a horizontal slab of air with thickness $dz$.
To be at rest, the pressure $P(z)$ supporting this slab from below must balance the pressure $P(z + dz)$ from above plus the weight of the slab itself.
Use this and the ideal gas law to derive the \textbf{barometric equation},
\begin{equation*}
  \frac{\partial P}{\partial z} = f(T) \, P,
\end{equation*}
and obtain an expression for $f(T)$.

This equation is easy to solve for $P(z)$ if we assume that the temperature of the air, $T$, is independent of the altitude $z$.
This is not necessarily a great assumption.
To test it, compute the relative atmospheric pressure at the top of Mount Everest (8850~m) compared to Liverpool (70~m), taking $T = 15${\textdegree}C and using $m = 4.811\times 10^{-26}$~kg as the average mass of an air molecule, along with the unit conversion factor $k_B \approx 1.381\times 10^{-23}$~J$\cdot$K$^{-1}$.
The true value is approximately 33.6\%.

\end{document}
% ------------------------------------------------------------------
