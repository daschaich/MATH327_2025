% ------------------------------------------------------------------
\documentclass[12 pt]{article} % A4 paper set by geometry package below
\pagenumbering{arabic}
\setlength{\parindent}{10 mm}
\setlength{\parskip}{12 pt}

% Nimbus Sans font should be reasonably legible
\usepackage{helvet}
\renewcommand{\familydefault}{\sfdefault}
\usepackage[T1]{fontenc}  % Without this \textsterling produces $

% Section header spacing
\usepackage{titlesec}
\titlespacing\section{0pt}{12pt plus 4pt minus 2pt}{0pt plus 2pt minus 2pt}
\titlespacing\subsection{0pt}{12pt plus 4pt minus 2pt}{0pt plus 2pt minus 2pt}
\titlespacing\subsubsection{0pt}{12pt plus 4pt minus 2pt}{0pt plus 2pt minus 2pt}

\usepackage{amsmath}
\usepackage{amssymb}
\usepackage{graphicx}
\usepackage{verbatim}    % For comment
\usepackage[shortlabels]{enumitem}
\usepackage[paper=a4paper, marginparwidth=0 cm, marginparsep=0 cm, margin=2.5 cm, includemp]{geometry}
\usepackage[pdftex, pdfstartview={FitH}, pdfnewwindow=true, colorlinks=true, citecolor=blue, filecolor=blue, linkcolor=blue, urlcolor=blue, pdfpagemode=UseNone]{hyperref}

% Put module code and last-modified date in footer
\usepackage{fancyhdr}
\pagestyle{fancy}
\fancyhf{}
\renewcommand{\headrulewidth}{0pt}
\cfoot{{\small \thisunit}\hfill \thepage\hfill {\small \moddate}}

% Hopefully address Canvas complaints about pdf tagging
%\usepackage[tagged]{accessibility}
\hypersetup {
  pdfauthor={David Schaich},
  pdftitle={StatMech Extra Practice},
}

% Shortcuts
\newcommand{\al}{\ensuremath{\alpha} }
\newcommand{\be}{\ensuremath{\beta} }
\newcommand{\ga}{\ensuremath{\gamma} }
\newcommand{\ka}{\ensuremath{\kappa} }
\renewcommand{\d}[1]{\ensuremath{\mathop{d#1}} }
\newcommand{\vev}[1]{\ensuremath{\left\langle #1 \right\rangle} }
% ------------------------------------------------------------------



% ------------------------------------------------------------------
\begin{document}
\newcommand{\thisunit}{MATH327 Extra (2d fermions)}
\newcommand{\moddate}{Last modified 12 May 2025}
\begin{center}
  {\Large \textbf{MATH327: StatMech and Thermo, Spring 2025}} \\[12 pt]
  {\Large \textbf{Extra practice \ --- \ 2d fermions}} \\[24 pt]
\end{center}

Consider a gas of fermions with mass $m$ confined to a two-dimensional surface of area $A$, with temperature $T = 1 / \be$ and chemical potential $\mu$.
Its grand-canonical potential is
\begin{equation*}
  \Phi(T, \mu) = -\frac{mAT}{2\pi\hbar^2} \int_0^{\infty} \log\left[1 + e^{-(E - \mu) / T}\right] \d{E}.
\end{equation*}

\begin{enumerate}[label={(\alph*)}]
  \item Show that the particle density is
        \begin{align*}
          \frac{\vev{N}}{A} & = \al \int_0^{\infty} F(E) \d{E}, \qquad &
          F(E) = \frac{1}{e^{(E - \mu) / T} + 1},
        \end{align*}
        where the constant $\al = \frac{m}{2\pi\hbar^2}$ and $F(E)$ is the usual Fermi function.

  \item Evaluate the particle density in the low-temperature limit where $F(E)$ becomes a step function.
        Use the result to find the Fermi energy $E_F = \mu$.

  \item Starting from the internal energy density
        \begin{equation*}
          \frac{\vev{E}}{A} = \frac{m}{2\pi\hbar^2} \int_0^{\infty} E \; F(E) \d{E},
        \end{equation*}
        show $\vev{E} = \ga \vev{N}^2$ in the low-temperature limit, and find the constant~$\ga$.

  \item The pressure is now the force per unit length on the boundary of the area $A$.
        In the low-temperature limit it is given by
        \begin{equation*}
          P = -\left. \frac{\partial \!\vev{E}}{\partial A}\right|_N.
        \end{equation*}
        Calculate $P$ to show $P A = \ka \vev{N} E_F$ in the low-temperature limit, and determine the constant $\ka$.
\end{enumerate}
\end{document}
% ------------------------------------------------------------------
