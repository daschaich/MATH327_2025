% ------------------------------------------------------------------
\documentclass[12 pt]{article} % A4 paper set by geometry package below
\pagenumbering{arabic}
\setlength{\parindent}{10 mm}
\setlength{\parskip}{12 pt}

% Nimbus Sans font should be reasonably legible
\usepackage{helvet}
\renewcommand{\familydefault}{\sfdefault}
\usepackage[T1]{fontenc}  % Without this \textsterling produces $

% Section header spacing
\usepackage{titlesec}
\titlespacing\section{0pt}{12pt plus 4pt minus 2pt}{0pt plus 2pt minus 2pt}
\titlespacing\subsection{0pt}{12pt plus 4pt minus 2pt}{0pt plus 2pt minus 2pt}
\titlespacing\subsubsection{0pt}{12pt plus 4pt minus 2pt}{0pt plus 2pt minus 2pt}

\usepackage{amsmath}
\usepackage{amssymb}
\usepackage{graphicx}
\usepackage{verbatim}    % For comment
\usepackage[shortlabels]{enumitem}
\usepackage[paper=a4paper, marginparwidth=0 cm, marginparsep=0 cm, margin=2.5 cm, includemp]{geometry}
\usepackage[pdftex, pdfstartview={FitH}, pdfnewwindow=true, colorlinks=true, citecolor=blue, filecolor=blue, linkcolor=blue, urlcolor=blue, pdfpagemode=UseNone]{hyperref}

% Put module code and last-modified date in footer
\usepackage{fancyhdr}
\pagestyle{fancy}
\fancyhf{}
\renewcommand{\headrulewidth}{0pt}
\cfoot{{\small \thisunit}\hfill \thepage\hfill {\small \moddate}}

% Hopefully address Canvas complaints about pdf tagging
%\usepackage[tagged]{accessibility}
\hypersetup {
  pdfauthor={David Schaich},
  pdftitle={StatMech Extra Practice},
}

% Shortcuts
\newcommand{\be}{\ensuremath{\beta} }
\newcommand{\eps}{\ensuremath{\varepsilon} }
\newcommand{\vev}[1]{\ensuremath{\left\langle #1 \right\rangle} }
\newcommand{\pderiv}[2]{\ensuremath{\frac{\partial #1}{\partial #2}} }
% ------------------------------------------------------------------



% ------------------------------------------------------------------
\begin{document}
\newcommand{\thisunit}{MATH327 Extra (Magnetization)}
\newcommand{\moddate}{Last modified 12 May 2025}
\begin{center}
  {\Large \textbf{MATH327: StatMech and Thermo, Spring 2025}} \\[12 pt]
  {\Large \textbf{Extra practice \ --- \ Magnetization}} \\[24 pt]
\end{center}

Consider a classical system of $N$ distinguishable, non-interacting `spins' in a lattice at temperature $T = 1 / \be$, where the value $s_n$ of each spin can vary \emph{continuously} in the range $-1 \leq s_n \leq 1$.
In an external magnetic field of strength $H > 0$, the internal energy of the system is $\displaystyle E = -H \sum_{n = 1}^N s_n$.

\begin{enumerate}[label={(\alph*)}]
  \item By integrating over the continuous $s_n$, calculate the canonical partition function $Z$ and the Helmholtz free energy $F$ of the system, both as functions of $\be H$.

  \item The derivative of the Helmholtz free energy with respect to the magnetic field defines the magnetization
        \begin{equation*}
          \vev{m} = -\frac{1}{N} \pderiv{F}{H}.
        \end{equation*}
        Assuming finite $H > 0$, calculate $\vev{m}$ for this system as a function of $\be H$, and determine its low- and high-temperature limits, $\displaystyle \lim_{T \to 0} \vev{m}$ and $\displaystyle \lim_{T \to \infty} \vev{m}$.

  \item For low but non-zero temperatures, expand $\vev{m}$ in terms of $\eps \equiv e^{-\be H} \ll 1$ to find the largest temperature-dependent term in this expansion.

  \item For high but finite temperatures, expand $\vev{m}$ in terms of $x \equiv \be H \ll 1$ to find the largest temperature-dependent term in this expansion.
\end{enumerate}
\end{document}
% ------------------------------------------------------------------
